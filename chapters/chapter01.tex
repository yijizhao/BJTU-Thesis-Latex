\setlength{\baselineskip}{20pt}
\chapter{引言}
\label{cha:intro}

\textbf{第二章和第三章已总结了笔者撰写论文过程中使用的技巧及遇到的问题解决方案,建议论文撰写者仔细阅读。}

[鼠标左键单击选择该段落,输入替换之。内容为小四号宋体。] 引言(或绪论)简要说明研究工作的目的、范围、相关领域的前人工作和知识空白、理论基础和分析、研究设想、研究方法和实验设计、预期结果和意义等。应言简意赅,不要与摘要雷同,不要成为摘要的注释。一般教科书中有的知识,在引言中不必赘述。


学位论文为了需要反映出作者确已掌握了坚实的基础理论和系统的专门知识,具有开阔的科学视野,对研究方案作了充分论证,因此,有关历史回顾和前人工作的综合评述,以及理论分析等,可以单独成章,用足够的文字叙述。正文是学位论文的核心部分,占主要篇幅,可以包括:调查对象、实验和观测方法、仪器设备、材料原料、实验和观测结果、计算方法和编程原理、数据资料、经过加工整理的图表、形成的论点和导出的结论等。


由于研究工作涉及的学科、选题、研究方法、工作进程、结果表达方式等有很大的差异,对正文内容不能作统一的规定。但是,必须实事求是,客观真切,准确完备,合乎逻辑,层次分明,简练可读。


\textbf{《北京交通大学学位论文撰写规范》:https://gs.bjtu.edu.cn/cms/item/477.html}


\section{2级标题}

命令:\textbackslash section\{2级标题\}


\subsection{3级标题}

命令:\textbackslash subsection\{3级标题\}