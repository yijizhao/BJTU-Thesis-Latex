 \setlength{\baselineskip}{20pt}
\chapter{参考文献格式说明}
\label{cha:chap3}


\section{参考文献格式说明}

\textbf{命令:}\textbackslash cite\{xxx\},如深度学习\cite{lecunDeepLearning2015}。

\textbf{注意:}中文文献一定要在bib文件中添加语言标识language=\{zh\},否则作者名字无法正常以中文形式的‘等’进行省略,参考下面的例子\cite{zhou2021}:

@article\{zhou2021,
	
	\indent\indent title = \{图神经网络驱动的交通预测技术:探索与挑战\},
	
	\indent\indent author = \{周毅 and 胡姝婷 and 李伟 and 承楠 and 路宁 and 沈学民\},
	
	\indent\indent year = \{2021\},
	
	\indent\indent journal =\{物联网学报\},
	
	\indent\indent volume = \{5\},
	
	\indent\indent number = \{04\},
	
	\indent\indent pages = \{1--16\},
	
	\indent\indent issn = \{2096-3750\},
	
	\indent\indent \textbf{language=\{zh\}}
	
\}


\textbf{参考文献检查项}:论文完成后应仔细检查参考文献格式,下面总结了一些在论文审阅过程中常被指出的问题:

\begin{itemize}
	\item 参考文献格式参照下一小节
	\item 除Arxiv文章,所有文献必须有页码、年份,期刊论文还应有卷号、期号
	\item 检查同一会议/期刊的名称是否统一了
	\item 检查文献标题每个单词是否均首字母大写
	\item 检查发表会议/期刊名称每个单词是否均首字母大写
	\item 检查是否有重复文献
\end{itemize}








\subsection{参考文献格式}

参考文献是文中引用的有具体文字来源的文献集合。按照GB 7714《文后参考文献著录规则》的规定执行。参考文献的格式为:

著作:[序号]作者.译者.书名[M].版本(第一版不著录).出版地:出版社,出版时间:引用部分起止页.

期刊:[序号]作者.译者.文章题目[J].期刊名,年份,卷号(期数):引用部分起止页.

会议论文集:[序号]作者.译者.文章名[C]. //编者.论文集名,会议地址,会议时间.出版地:出版者,出版年.引用部分起止页.

学位论文:[序号]作者.题名[D].保存地点:保存单位,年份.引用部分起止页.

专利:[序号]专利申请者.专利文献题名[P].国别,专利文献种类,专利号.发布日期:引用部分起止页.

技术标准:[序号]起草责任者.标准代号.标准顺序号——发布年.标准名称.出版地.出版者.出版年份:引用部分起止页.

报纸: [序号]作者.题名[N].报纸名,出版日期(版次)


